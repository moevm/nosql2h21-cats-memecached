\pagebreak
\begin{center}
    \textbf{АННОТАЦИЯ}
\end{center}

В данном проекте было разработано веб-приложение для просмотра и фильтрации списка
пород кошек.
Реализация приложения предполагала одно ограничение -- в качестве СУБД должен
использоваться Memcached.

В процессе работы был разработан макет пользовательского интерфейса,
анализ основных сценариев использования приложения конечным пользователем.
Также была построена модель данных с учетом ограничений выбранного инструмента
по работе с данными.
Был проведен сравнительный анализ решения на основе Memcached и решения на основе
реляционной БД Postgres по объему хранения данных и количеству запросов для CRUD операций.

Клиентская часть приложения была написана с использованием библиотеки React,
серверная часть разработана на Java с использованием фреймворка Spring.
Разработанное приложение может запущено с помощью средств контейнеризации и оркестрации,
в частности docker и docker-compose.